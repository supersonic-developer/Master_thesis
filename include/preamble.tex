%--------------------------------------------------------------------------------------
% Page layout setup
%--------------------------------------------------------------------------------------
% we need to redefine the pagestyle plain
% another possibility is to use the body of this command without \fancypagestyle
% and use \pagestyle{fancy} but in that case the special pages
% (like the ToC, the References, and the Chapter pages)remain in plane style
% chktex 7

\pagestyle{plain}
\geometry{inner=35mm, outer=25mm, top=28mm, bottom=25mm}
\setstretch{1.5}

\setcounter{tocdepth}{3}
%\sectionfont{\large\upshape\bfseries}
\setcounter{secnumdepth}{3}

\sloppy % Margón túllógó sorok tiltása.
\widowpenalty=10000 \clubpenalty=10000 %A fattyú- és árvasorok elkerülése
\def\hyph{-\penalty0\hskip0pt\relax} % Kötőjeles szavak elválasztásának engedélyezése


%--------------------------------------------------------------------------------------
% Setup hyperref package
%--------------------------------------------------------------------------------------
\hypersetup{
    % bookmarks=true,            % show bookmarks bar?
    unicode=true,              % non-Latin characters in Acrobat's bookmarks
    pdftitle={\vikcim},        % title
    pdfauthor={\szerzoMeta},    % author
    pdfsubject={\vikdoktipus}, % subject of the document
    pdfcreator={\szerzoMeta},   % creator of the document
    pdfproducer={},    % producer of the document
    pdfkeywords={},    % list of keywords (separate then by comma)
    pdfnewwindow=true,         % links in new window
    colorlinks=true,           % false: boxed links; true: colored links
    linkcolor=black,           % color of internal links
    citecolor=red,           % color of links to bibliography
    filecolor=black,           % color of file links
    urlcolor=blue             % color of external links
}

\def\UrlBreaks{\do-\do_}
%--------------------------------------------------------------------------------------
% Set up listings
%--------------------------------------------------------------------------------------
\definecolor{lightgray}{rgb}{0.95,0.95,0.95}
\lstset{
	basicstyle=\scriptsize\ttfamily, % print whole listing small
	backgroundcolor=\color{lightgray},
	columns=flexible,
	keepspaces=true,
	captionpos=b,
	breaklines=true,
	frame=single,
	tabsize=2,
	aboveskip=0.5cm,
	belowskip=0.3cm,
	literate=*
        {á}{{\'a}}1	{é}{{\'e}}1	{í}{{\i'}}1	{ó}{{\'o}}1	{ö}{{\"o}}1	{ő}{{\H{o}}}1	{ú}{{\'u}}1	{ü}{{\"u}}1	{ű}{{\H{u}}}1
		{Á}{{\'A}}1	{É}{{\'E}}1	{Í}{{\'I}}1	{Ó}{{\'O}}1	{Ö}{{\"O}}1	{Ő}{{\H{O}}}1	{Ú}{{\'U}}1	{Ü}{{\"U}}1	{Ű}{{\H{U}}}1
}

\definecolor{keywordplum}{RGB}{130, 20, 100}
\lstdefinestyle{STM32Cube}
{
    language=C,
    commentstyle=\color{OliveGreen},
    stringstyle=\color{Blue},
    numbers=left,
    emph={if, else, return, sizeof, void, U32, switch, case, break, default},
    emphstyle=\bfseries\color{keywordplum}\textbf,
    emph=[2]{UART_Flag_RxReady, validBuffer, otherData, dataLen, header, dataID, rawEXBusData, len, huart, RxState, HAL_UART_STATE_READY, gState, receivedByte, UART_Status},
    emphstyle=[2]\color{blue}\textit,
    emph=[3]{EX_BUS_Handler},
    emphstyle=[3]\color{black}\textbf
}

\definecolor{namespacecolor}{rgb}{0.3059, 0.7882, 0.6902}
\lstdefinestyle{VS2019}
{
    language=C,
    stringstyle=\color{red},
    showstringspaces=false,
    commentstyle=\color{OliveGreen},
    numbers=left,
    emph={if, else, return, for},
    emphstyle=\color{keywordplum}\textbf,
    emph=[2]{public, bool, byte, ref, int, true, false, uint, void, double},
    emphstyle=[2]\color{blue},
    emph=[3]{I2C_Receive, I2C_Transmit, I2C_Calculate_FirstByte, I2C_SendByteCheckACK, I2C_Create_StopCondition, I2C_ReadByteAndSendACKOrNACK, I2C_Create_StartingCondition, Send_Data, Receive_Data, Sleep, WriteLine, Determine_Addr_Name,CopyData_fromJsonFile_intoBuffer, Write_ServoData, Write, UploadProgram, RepresentMemory_ofHexFile, CreateBinaryData_fromMemRepresentation, Write_RequestedNumBytes, BlockCopy},
    emphstyle=[3]\color{Apricot},
    emph=[4]{Thread, Console, Buffer},
    emphstyle=[4]\color{namespacecolor},
}

\lstdefinestyle{linker-script}
{
    basicstyle=\scriptsize\ttfamily\color{blue},
	backgroundcolor=\color{lightgray},
	columns=flexible,
	keepspaces=true,
	captionpos=b,
	breaklines=true,
	frame=single,
	tabsize=2,
	aboveskip=0.5cm,
	belowskip=0.3cm,
    numbers=left,
    numberstyle=\color{black},
    moredelim=**[is][\color{black}]{@}{@},
    moredelim=**[is][\color{red}]{<}{>}
}


\definecolor{stringcolor}{rgb}{0.8392, 0.6157, 0.5216}
\lstdefinestyle{cpputest}
{
    language=C,
    commentstyle=\color{OliveGreen},
    stringstyle=\color{red},
    numbers=left,
    emph=[2]{extern, void, sizeof, int},
    emphstyle=[2]\color{blue}\textbf,
    emph={TEST, TEST_GROUP, setup, teardown},
    emphstyle=\color{stringcolor}\textbf,
    emph=[3]{if, else, return, for},
    emphstyle=[3]\color{keywordplum}\textbf,
}

%--------------------------------------------------------------------------------------
% Set up theorem-like environments
%--------------------------------------------------------------------------------------
% Using ntheorem package -- see http://www.math.washington.edu/tex-archive/macros/latex/contrib/ntheorem/ntheorem.pdf

\theoremstyle{plain}
\theoremseparator{.}
\newtheorem{example}{\pelda}

\theoremseparator{.}
%\theoremprework{\bigskip\hrule\medskip}
%\theorempostwork{\hrule\bigskip}
\theorembodyfont{\upshape}
\theoremsymbol{{\large \ensuremath{\centerdot}}}
\newtheorem{definition}{\definicio}

\theoremseparator{.}
%\theoremprework{\bigskip\hrule\medskip}
%\theorempostwork{\hrule\bigskip}
\newtheorem{theorem}{\tetel}


%--------------------------------------------------------------------------------------
% Some new commands and declarations
%--------------------------------------------------------------------------------------
\newcommand{\code}[1]{{\upshape\ttfamily\scriptsize\indent #1}}
\newcommand{\doi}[1]{DOI:\@ \href{http://dx.doi.org/\detokenize{#1}}{\raggedright{\texttt{\detokenize{#1}}}}} % A hivatkozások közt így könnyebb DOI-t megadni.

\DeclareMathOperator*{\argmax}{arg\,max}
%\DeclareMathOperator*[1]{\floor}{arg\,max}
\DeclareMathOperator{\sign}{sgn}
\DeclareMathOperator{\rot}{rot}

\sisetup{per-mode=symbol,per-symbol = /}


%--------------------------------------------------------------------------------------
% Figures settings
%--------------------------------------------------------------------------------------
\captionsetup{aboveskip=0.1cm, belowskip=0.1cm}
\setlength{\intextsep}{0.2cm}
\setlength{\textfloatsep}{0.2cm}

\renewcommand{\captionlabelfont}{\bf}
%\renewcommand{\lstlistingname}{Forráskódrészlet}
%\renewcommand{\lstlistlistingname}{Forráskódrészlet-jegyzék}

\titleformat{\chapter}[display]
{\Huge\bfseries\scshape\raggedright}
{\chaptertitlename\ \thechapter}{0cm}
{}[\titlerule]



\titlespacing*{\chapter}{0pt}{0.6cm}{0.3cm}
\titlespacing*{\section}{0pt}{0.4cm}{0.3cm}
\titlespacing*{\subsection}{0pt}{0.2cm}{0.3cm}
\author{\vikszerzo}
\title{\viktitle}

%\makeatletter
%\patchcmd{\chapter}{\if@openright\cleardoublepage\else\clearpage\fi}{}{}{}
%\makeatother

\allowdisplaybreaks%
