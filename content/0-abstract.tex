\selecthungarian%
\chapter*{Kivonat}\addcontentsline{toc}{chapter}{Kivonat}

Az elmúlt évtizedekben a pilóta nélküli légijárművek (Unmanned Aerial Vehicle, UAV) egyre népszerűbbé váltak civil és akadémiai alkalmazásokban is, mint például felderítés, áruszállítás, geofizikai adatgyűjtés, mentési műveletek vagy éppen mezőgazdasági célú felhasználás. Az előrelépés nemcsak a technikai modernizációnak köszönhető, hanem a szélesebb körű felhasználhatóságnak, ami nagyban a hagyományos navigációs módszerek innovációjának köszönhető. A UAV-k jövőre gyakorolt hatása azon is múlik mennyire tudnak jól navigálni GPS (Global Positioning System) nélküli környezetben például kimaradó, zavart (jamming) vagy hamisított (spoofing) jelek esetén.

Dolgozatom középpontjában egy vizuális-inerciális navigációs algoritmus áll. A tanulmány során megvalósított rendszer célja meghatározni egy légijármű pozícióját, orientációját, sebességét és a gyorsulás és szögsebesség szenzor bias értékeit inerciális szenzorrendszer (Inertial Measurement Unit, IMU) mérések és mono kameraképek alapján. A javasolt rendszer egy hiba állapot Kálmán-szűrő (Error-State Kalman Filter, ESKF) alapú keretrendszerben hajtja végre az IMU és a kamera adatok integrációját. Az IMU mérésekből a repülőgép pozícióját, sebességét és orientációját lehet becsülni, amelynek hibáját a kamera képekből nyert információval korrigálom. A korrekció során felhasznált jellegpontok pozícióját háromszögelés útján számítom, amelyhez egy ún. Linear Optimal Sine Triangulation (LOST) módszert alkalmazok. Az algoritmus tesztelése úgy történik, hogy a kezdeti állapotban ismert GPS koordinátákat feltételezek, és ez idő alatt már megkezdődik a jellegpontok pozíciójának optimalizációja, amelyeket az ESKF frissítési fázisában használok fel. Később a GPS koordináták már nem ismertek, ezért az ESKF becslésekből történik az újabb jellegpontok pozíciójának optimalizációja, azok alapján pedig a frissítés. Munkámban megvizsgálom a módszer pontosságának korlátait, és azt is, hogy a GPS jel elvesztése után meddig ad kielégítő pontosságot csak a vizuális-inerciális navigáció.

Összefoglalva, a dolgozat bemutatja az algoritmus fejlesztését, amely fuzionálja az inerciális és vizuális adatokat, valamint az ehhez szükséges elméleti alapismereteket és matematikai módszertanokat. Az alap algoritmusokat (ESKF és LOST) szakirodalomból vettem, amelyeket saját rendszerbe integráltam. A kutatás során először egy szimulációt hoztam létre, amelyet fokozatosan közelítettem a valóságos körülményeket leginkább modellező környezethez. Az ígéretes szimulációs eredményeket követően fő célom az elkészített navigációs rendszer tesztelése lesz szintetikus vagy valós felvételeken.


\selectthesislanguage%
\chapter*{Abstract}\addcontentsline{toc}{chapter}{Abstract}

Over the past decades, Unmanned Aerial Vehicles (UAVs) have become increasingly popular in both civilian and academic applications, such as exploration, cargo delivery, geophysical data collection, rescue operations, and agricultural purposes. This expansion required not only technical modernization but also advancement in traditional navigation methods. The future impact of UAVs also depends on their ability to navigate effectively in GPS (Global Positioning System)-denied environments, for example, scenarios involving signal dropout, jamming, or spoofing.

My work focuses on a visual-inertial navigation algorithm. In this study, the implemented system's goal is to determine an aircraft's position, orientation, velocity, and bias values of the accelerometer and gyroscope based on measurements from an Inertial Measurement Unit (IMU) and monocular camera images. The proposed system performs the integration of IMU and camera data within an Error-State Kalman Filter (ESKF) framework. The aircraft's position, velocity, and orientation are estimated from IMU measurements, and these estimates are corrected with the information obtained from camera images. During the correction, the positions of feature points are computed through triangulation using a method called Linear Optimal Sine Triangulation (LOST). The algorithm is tested as follows: in the initial stage the GPS coordinates are assumed to be known, and the optimization of feature point positions begins this time, and they are utilized in the ESKF update. Later, when GPS coordinates are no longer known, the optimization of feature point positions is based on the ESKF estimates, and these optimized values are used in the update. In my work, I examine the limitations of the method's accuracy and investigate how long it can provide satisfactory accuracy after losing the GPS signal, relying only on visual-inertial navigation.

To summarize, the thesis presents the development of an algorithm that fuses inertial and visual data besides the required theoretical background and mathematical foundations. The utilized algorithms (ESKF and LOST) were chosen from the literature and integrated into the newly developed system. I applied a simulation environment and a gradual approach during the development process, starting with ideal conditions and incrementally introducing more realistic elements to the simulation. After achieving promising simulation results, my primary objective in the future is to test the developed navigation system on synthetic or real-world footage.